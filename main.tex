\documentclass[12pt,a4paper,openright,dvipsnames]{mwrep}

%%% libraries

% ~egreg | https://tex.stackexchange.com/a/95842

\usepackage{array, tabularx}

\newenvironment{conditions}
  {\par\vspace{\abovedisplayskip}\noindent
   \begin{tabular}{>{$}l<{$} @{} >{${}}c<{{}$} @{} l}}
  {\end{tabular}\par\vspace{\belowdisplayskip}}

\newenvironment{conditions*}
  {\par\vspace{\abovedisplayskip}\noindent
   \tabularx{\columnwidth}{>{$}l<{$} @{}>{${}}c<{{}$}@{} >{\raggedright\arraybackslash}X}}
  {\endtabularx\par\vspace{\belowdisplayskip}} % where: conditions
\input{lib/karnaugh.tex} % karnaugh maps


%%% packages and basic settings

\usepackage{lmodern} % fixes poor font quality (use only if NOT using XeLaTeX or LuaLaTeX)
\usepackage[polish]{babel}
\usepackage[T1]{polski}
\usepackage[utf8]{inputenc}

\usepackage[a4paper,
            tmargin=2cm,
            bmargin=2cm,
            lmargin=2cm,
            rmargin=2cm,
            bindingoffset=0cm]{geometry}

% ToC and hyper links
\usepackage{tocloft} % controlling the typographic design of the Table of Contents
\usepackage{hyperref} % handle cross-referencing commands to produce hypertext links in the document

% maths
\usepackage{amsmath} % improves the information structure and printed output
\usepackage{amssymb} % provides an extended symbol collection
\usepackage{siunitx} % units \SI \si
% \usepackage{amsthm} % helps to define theorem-like structures

% code syntax
\usepackage{minted} % syntax highlighting using pygments (check docs for changing themes and styling)

% graphics and styling
\usepackage{graphicx} % graphics support
\usepackage{subfig}
\usepackage{float} % improves the interface for defining floating objects
\usepackage{setspace} % sets space between lines

% tables (check docs for every packages as each has a different task)
\usepackage{booktabs}
\usepackage{array}
\usepackage{makecell}
\usepackage{tabularx}

% coloring utility
\usepackage{xcolor}

% advanced graphic stuff
\usepackage{tikz}
\usetikzlibrary{automata, positioning, arrows} % automata diagrams

% black hyperlinks (change to blue for highlighting)
\hypersetup{
    colorlinks,
    citecolor=black,
    filecolor=black,
    linkcolor=black,
    urlcolor=black
}


%%% variables

\newcommand{\vtitle}{Title}
\newcommand{\vauthors}{
    Author One\\
    Author Two
}



%%% code starts here


% title page AGH
\begin{document}
\begin{titlepage}

    \centering
    \singlespacing

    \includegraphics[scale=0.5]{common/agh.png}

    \vspace*{0.4cm}

    \fontsize{12}{16}{\textbf{\MakeUppercase{Akademia Górniczo-Hutnicza}}}\\
    \fontsize{10}{12}{\textbf{\MakeUppercase{Im. Stanisława Staszica w Krakowie}}}

    \vspace*{0.4cm}

    \vspace{\stretch{0.3}}

    {\LARGE\bfseries \vtitle\\}
    \rule{3in}{0.4pt} \\
    \large\vauthors


    \vspace*{\stretch{2}}

    \centerline{\rmfamily\fontsize{14}{16}\selectfont Kraków, \today}

\end{titlepage}
\newpage

% ToC
\tableofcontents
\newpage


%
\chapter{Rozdział}

%%
\section{Sekcja 1}

%%%
\subsection{Syntax highlight}

\phantomsection\hypertarget{python-code}{}
\begin{minted}{python}
def getFirstPrimeNumbers(n):
    """Returns a list of the first n prime numbers"""
    primes = []
    i = 2
    while len(primes) < n:
        if isPrime(i):
            primes.append(i)
        i += 1
    return primes



def isPrime(n):
    """Returns True if n is prime, False otherwise"""
    for i in range(2, n):
        if n % i == 0:
            return False
    return True

if __name__ == "__main__":
    print(getFirstPrimeNumbers(10))
\end{minted}
\newpage

%%%
\subsection{Images}

\begin{figure}[H] % forces figure to this position (does not allow LaTeX to relocate for best fit)
    \centering
    \includegraphics[width=\textwidth]{example-image-a}
    \caption{example A image}
    \label{fig:example-a}
\end{figure}

\begin{figure}[H]
    \centering
    \subfloat[example subfig B]{%
        \includegraphics[width=0.5\textwidth]{example-image-b}%
    }
    \subfloat[example subfig C]{%
        \includegraphics[width=0.5\textwidth]{example-image-c}%
    }
    \caption{Subfigures B and C}
\end{figure}


%%%
\subsection{Phantom hyperlink}

Znalazłem interesujący \hyperlink{python-code}{kod} rozwiązujący nasz problem.


%%%
\subsection{Conditions}

Short conditions:
\begin{equation}
    \rho S \Delta x
    \frac{\partial^2 \Psi(x, t)}{\partial t^2}
    =
    S(\sigma (x + \Delta x) - \sigma(x))
    \text{,}
\end{equation}
gdzie:
\begin{conditions}
    S & - & powierzchnia przekroju pręta, \\
    \rho & - & gęstość materiału, \\
    \sigma & - & naprężenie, \\
    \Psi(x,t) & - & średnie przemieszczenie wycinka pręta.
\end{conditions}

\vspace*{0.6cm}

Long conditions:
\begin{equation}
    \rho S \Delta x
    \frac{\partial^2 \Psi(x, t)}{\partial t^2}
    =
    S(\sigma (x + \Delta x) - \sigma(x))
    \text{,}
\end{equation}
gdzie:
\begin{conditions*}
    S & - & powierzchnia przekroju pręta, którego zamierzamy obliczyć
            w dalszej części zadania, \\
    \rho & - & gęstość materiału, \\
    \sigma & - & naprężenie, \\
    \Psi(x,t) & - & średnie przemieszczenie wycinka pręta.
\end{conditions*}


%%%
\subsection{Automata diagram}

\begin{figure}[H]
    \tikzset{
        ->, % makes the edges directed
        >=stealth', % makes the arrow heads bold
        node distance=3cm, % specifies the minimum distance between two nodes. Change if necessary.
        every state/.style={thick, fill=gray!10}, % sets the properties for each ’state’ node
        initial text=$ $, % sets the text that appears on the start arrow
    }
    \centering
    \begin{tikzpicture}
        \node[state, initial] at (-1.763, 2.427)     (q1)      {0000};
        \node[state] at (1.763, 2.427)      (q2)      {0001};
        \node[state] at (2.853, -0.927)     (q3)      {0011};
        \node        at (0, -3)       (q4)  {$\cdots$}; 
        \node[state] at (-2.853, -0.927)    (q5)      {1000};



        \draw   (q1)    edge[bend left, above]  (q2)
                (q2)    edge[bend left, above]  (q3)
                (q3)    edge[bend left, above]  (q4)
                (q4)    edge[bend left, above]  (q5)
                (q5)    edge[bend left, above]  (q1); % A 0 lopp
                % (q1)    edge[bend left, above]  node{1} (q2) % A 1 to B

                % (q2)    edge[bend left, below]  node{0} (q1) % B 0 to A
                % (q2)    edge[right]             node{1} (q3) % B 1 to C

                % (q3)    edge[loop below]        node{1} (q3) % C 1 loop
                % (q3)    edge[above]             node{0} (q4) % C 0 to D

                % (q4)    edge[above]             node{1} (q5) % D 1 to E
                % (q4)    edge[right]             node{0} (q1) % D 0 to A
                
                % (q5)    edge[above]             node{0} (q1); % E 0 to A




    \end{tikzpicture}
    \caption{Schemat automatu}
\end{figure}


\begin{figure}[H]
    \tikzset{
        ->, % makes the edges directed
        >=stealth', % makes the arrow heads bold
        node distance=3cm, % specifies the minimum distance between two nodes. Change if necessary.
        every state/.style={thick, fill=gray!10}, % sets the properties for each ’state’ node
        initial text=$ $, % sets the text that appears on the start arrow
    }
    \centering
    \begin{tikzpicture}
        \node[state, initial] at (0, 0) (q1) {A};
        \node[state] at (4, 0) (q2) {B};
        \node[state, accepting] at (0, -4) (q5) {E};
        \node[state] at (4, -4) (q3) {C};
        \node[state] at (2, -2) (q4) {D};

        \draw   (q1)    edge[loop above]        node{0} (q1) % A 0 lopp
                (q1)    edge[bend left, above]  node{1} (q2) % A 1 to B

                (q2)    edge[bend left, below]  node{0} (q1) % B 0 to A
                (q2)    edge[right]             node{1} (q3) % B 1 to C

                (q3)    edge[loop below]        node{1} (q3) % C 1 loop
                (q3)    edge[above]             node{0} (q4) % C 0 to D

                (q4)    edge[below]             node{1} (q5) % D 1 to E
                (q4)    edge[bend left, below]  node{0} (q1) % D 0 to A
                
                (q5)    edge[left]              node{0} (q1) % E 0 to A
                (q5)    edge[below]             node{1} (q3); % E 1 to C




    \end{tikzpicture}
    \caption{Schemat automatu}
\end{figure}


%%%
\subsection{Karnaugh map}

\begin{figure}[H]
    \centering
    \begin{Karnaugh}{$Q_0 X$}{$Q_2 Q_1$}
        \contingut{0,1,0,0,1,0,0,0,0,0,-,-,-,-,-,-}
        \implicant{1}{1}{green}
        \implicant{4}{12}{red}
    \end{Karnaugh}
    \caption{Tabela Karnaugh dla bitu 0}
\end{figure}

\begin{figure}[H]
    \centering
    \begin{Karnaugh}{$Q_0 X$}{$Q_2 Q_1$}
        \contingut{0,0,0,1,1,1,0,0,0,1,-,-,-,-,-,-}
        \implicant{3}{3}{green}
        \implicant{4}{13}{red}
        \implicant{13}{11}{blue}
    \end{Karnaugh}
    \caption{Tabela Karnaugh dla bitu 1}
\end{figure}

\begin{figure}[H]
    \centering
    \begin{Karnaugh}{$Q_0 X$}{$Q_2 Q_1$}
        \contingut{0,0,0,0,0,0,0,1,0,0,-,-,-,-,-,-}
        \implicant{7}{15}{green}
    \end{Karnaugh}
    \caption{Tabela Karnaugh dla bitu 2}
\end{figure}

\end{document}
